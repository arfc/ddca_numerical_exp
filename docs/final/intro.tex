\section{Introduction}
\gls{NFC} simulation scenarios are constrained objective functions. 
The objectives are systemic demands such as "1\% power growth", 
while the constraints are availability of new nuclear technology.
To effectively simulate a nuclear fuel cycle, \gls{NFC} simulators 
must bring demand responsive deployment decisions into the dynamics
of the simulation logic \cite{huff_current_2017}. 

** Why? 

Thus, a \gls{NFC} simulator should have the capability to deploy 
supporting fuel cycle facilities to meet a user-defined commodity
demand. 
While automated power production deployment is common in most fuel
cycle simulators, automated deployment of supportive fuel cycle 
facilities is non-existent. 
Instead, the user must detail the deployment timeline of all 
supporting facilities or have infinite capacity support facilities
to ensure that there is no gap in the nuclear fuel cycle supply 
chain. 
These user-defined assumptions are not an accurate reflection 
of the real world. 
This shortcoming exists also in the fuel cycle simulator, \Cyclus. 
Therefore, there is a need to develop demand-driven deployment 
capability in \Cyclus to deploy facilities to meet front-end and 
back-end demands of the fuel cycle.

The Demand-Driven Cycamore Archetype project (NEUP-FY16-10512) 
aims to develop \Cyclus's demand-driven deployment capabilities. 
The developed algorithm will be in the form of a \Cyclus 
\texttt{Institution} agent, and will deploy \texttt{Facilities} 
to meet the front-end and back-end demands of the fuel cycle.
This demand-driven deployment capability is referred to as 
\deploy. 
Its goal is to meet demand for any commodity while minimizing 
oversupply. 

** Description of using \deploy to run transition scenarios more effectively ** 
