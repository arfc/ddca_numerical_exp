\documentclass[11pt,letterpaper]{article}
\usepackage[utf8]{inputenc}
\usepackage{caption} % for table captions
\usepackage{amsmath} % for multi-line equations and piecewises
\DeclareMathOperator{\sign}{sign}
\usepackage{graphicx}
\usepackage{relsize}
%\usepackage{textcomp}
\usepackage{xspace}
\usepackage{verbatim} % for block comments
%\usepackage{subfig} % for subfigures
\usepackage{enumitem} % for a) b) c) lists
\newcommand{\Cyclus}{\textsc{Cyclus}\xspace}%
\newcommand{\Cycamore}{\textsc{Cycamore}\xspace}%
\newcommand{\deploy}{\texttt{d3ploy}\xspace}%
\usepackage{tabularx}
\usepackage{color}
\usepackage[acronym,toc]{glossaries}
\include{acros}
\definecolor{bg}{rgb}{0.95,0.95,0.95}
\newcolumntype{b}{X}
\newcolumntype{f}{>{\hsize=.15\hsize}X}
\newcolumntype{s}{>{\hsize=.5\hsize}X}
\newcolumntype{m}{>{\hsize=.75\hsize}X}
\newcolumntype{r}{>{\hsize=1.1\hsize}X}
\usepackage{titling}
\usepackage[hang,flushmargin]{footmisc}
\renewcommand*\footnoterule{}
\usepackage[newfloat]{minted}
\newenvironment{code}{\captionsetup{type=listing}}{}
\SetupFloatingEnvironment{listing}{name=Code}
\newcolumntype{P}[1]{>{\centering\arraybackslash}p{#1}}

\bibliographystyle{abbrv}
\usepackage{tikz}


\usetikzlibrary{shapes.geometric,arrows}
\tikzstyle{process} = [rectangle, rounded corners, minimum width=1cm, minimum height=1cm,text centered, draw=black, fill=blue!30]
\tikzstyle{arrow} = [thick,->,>=stealth]


\graphicspath{{images/}}
\title{Final Report}
\author{Gwendolyn J. Chee, Roberto E. Fairhurst, Robert R. Flanagan, Kathryn D. Huff}


\begin{document}
	\maketitle
	\hrule


%----------------------------------------------------------------%
\section{Introduction}
\gls{NFC} simulation scenarios are constrained objective functions. 
The objectives are systemic demands such as "1\% power growth", 
while the constraints are availability of new nuclear technology.
To effectively simulate a nuclear fuel cycle, \gls{NFC} simulators 
must bring demand responsive deployment decisions into the dynamics
of the simulation logic \cite{huff_current_2017}. 

** Why? 

Thus, a \gls{NFC} simulator should have the capability to deploy 
supporting fuel cycle facilities to meet a user-defined commodity
demand. 
While automated power production deployment is common in most fuel
cycle simulators, automated deployment of supportive fuel cycle 
facilities is non-existent. 
Instead, the user must detail the deployment timeline of all 
supporting facilities or have infinite capacity support facilities
to ensure that there is no gap in the nuclear fuel cycle supply 
chain. 
These user-defined assumptions are not an accurate reflection 
of the real world. 
This shortcoming exists also in the fuel cycle simulator, \Cyclus. 
Therefore, there is a need to develop demand-driven deployment 
capability in \Cyclus to deploy facilities to meet front-end and 
back-end demands of the fuel cycle.

The Demand-Driven Cycamore Archetype project (NEUP-FY16-10512) 
aims to develop \Cyclus's demand-driven deployment capabilities. 
The developed algorithm will be in the form of a \Cyclus 
\texttt{Institution} agent, and will deploy \texttt{Facilities} 
to meet the front-end and back-end demands of the fuel cycle.
This demand-driven deployment capability is referred to as 
\deploy. 
Its goal is to meet demand for any commodity while minimizing 
oversupply. 

** Description of using \deploy to run transition scenarios more effectively ** 

%----------------------------------------------------------------%
\section{Background}
Hello I am \Cyclus. What am I? 

%----------------------------------------------------------------%
\section{Method}
To meet the project objective, an algorithm that deploys 
facilities to meet demand for any commodity while 
minimizing oversupply was developed. 
At each time step, the demand and supply for each commodity is 
predicted for the following time step. 
Based on the prediction, \deploy deploys facilities to meet the 
predicted demand. 
The demand and supply predictions are governed by four types
of algorithms: non-optimizing, time series forecasting, 
deterministic optimizing and machine learning. 
The choice of which prediction algorithm to use is a user input. 

\subsection{Non-Optimizing}
List and describe each algorithm 

\subsection{Time Series Forecasting}
List and describe each algorithm 

\subsection{Deterministic Optimizing}
List and describe each algorithm 

\subsection{Machine Learning}
List and describe each algorithm 

%----------------------------------------------------------------%
\section{Demonstration of \deploy capabilities}
The \deploy capabilities will be demonstrated through numerical
experiments. 
The numerical experiments are in the form of fuel cycle scenarios 
where the demand driving commodity, its demand curve and the 
combination of facilities in the scenario are varied. 
Each numerical experiment will be run for each prediction
algorithm. 
The prediction algorithms will be compared to determine each 
of their strengths and weaknesses. 
And how their overall performance demonstrates \deploy's 
capabilities. 

The basis of comparison for the numerical experiments are: 
the number of time steps where demand exceeds supply, residuals
and $\chi^2$ goodness of fit test. 

The numerical experiments are broken down into four types: 
front-end facility deployment, back-end facility deployment, 
closed fuel cycle and transition scenario. 

\subsection{Front-end Deployment}

\begin{table}[H]
	\centering
	\caption {Front-end Deployment Numerical Experiments}
	\label{tab:fenum}
	\begin{tabular}{|l|p{2.5cm}|p{2.5cm}|l|p{2.8cm}|}
		\hline
		\textbf{Test Scenario} & \textbf{Facilities Present} & \textbf{\shortstack{Reactor \\ Parameters}} & \textbf{\shortstack{Demand-driving \\ commodity}} & \textbf{\shortstack{Demand \\ Equation}}\\
		\hline
		1 & \texttt{Source} and \texttt{Sink} & - & fuel & 1000*t\\
		2 & \texttt{Source} and \texttt{Reactor} & Cycle time: 1, Refuel time: 0 & Power & 1000*t\\
		3 & \texttt{Source} and \texttt{Reactor} & Cycle time: 3, Refuel time: 1 & Power & 1000*t\\
		4 & \texttt{Source}, \texttt{Reactor} and \texttt{Sink} & Cycle time: 1, Refuel time: 0 & Power & 10*(1+1.5)**(t/12)\\
		\hline
	\end{tabular}
\end{table}

\subsection{Back-end Deployment}

\subsection{Closed Fuel Cycle}

\subsection{Transition Scenario}

\section{Transition Scenarios}


\pagebreak 
\bibliography{bibliography}

\end{document}


