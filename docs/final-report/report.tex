\documentclass[11pt]{article}
\usepackage[utf8]{inputenc} % Required for inputting international characters
\usepackage[T1]{fontenc} % Output font encoding for international characters
\usepackage{caption} % for table captions
\usepackage{amsmath} % for multi-line equations and piecewises
\DeclareMathOperator{\sign}{sign}
\usepackage{graphicx}
\usepackage{relsize}
\usepackage{xspace}
\usepackage{verbatim} % for block comments
\usepackage{subcaption} % for subfigures
\usepackage{enumitem} % for a) b) c) lists
\newcommand{\Cyclus}{\textsc{Cyclus}\xspace}%
\newcommand{\Cycamore}{\textsc{Cycamore}\xspace}%
\newcommand{\deploy}{\texttt{d3ploy}\xspace}%
\newcommand{\Deploy}{\texttt{D3ploy}\xspace}%
\usepackage{tabularx}
\usepackage{color}
\usepackage{multirow}
\usepackage[acronym,toc]{glossaries}
\include{acros}
\definecolor{bg}{rgb}{0.95,0.95,0.95}
\newcolumntype{b}{X}
\newcolumntype{f}{>{\hsize=.15\hsize}X}
\newcolumntype{s}{>{\hsize=.5\hsize}X}
\newcolumntype{m}{>{\hsize=.75\hsize}X}
\newcolumntype{r}{>{\hsize=1.1\hsize}X}
\usepackage{titling}
\usepackage[hang,flushmargin]{footmisc}
\renewcommand*\footnoterule{}
\usepackage{tikz}
\definecolor{illiniblue}{HTML}{437db6}
\definecolor{illiniorange}{HTML}{E38749}
\usetikzlibrary{shapes.geometric, arrows}
\tikzstyle{oblock} = [rectangle, draw, fill=illiniorange, 
text width=12em, text centered, rounded corners, minimum height=4em]
\tikzstyle{bblock} = [rectangle, draw, fill=illiniblue, 
text width=12em, text centered, rounded corners, minimum height=4em]
\tikzstyle{arrow} = [thick,->,>=stealth]

\usetikzlibrary{shapes.geometric,arrows}
\tikzstyle{process} = [rectangle, rounded corners, 
minimum width=1cm, minimum height=1cm,text centered, draw=black, 
fill=blue!30]
\tikzstyle{arrow} = [thick,->,>=stealth]

\graphicspath{}
\usepackage{mathpazo} % Palatino font
\usepackage{graphicx} % For the logo
\usepackage{float} 

\usepackage[hidelinks]{hyperref}
\urlstyle{tt}

\newcolumntype{L}{>{\raggedright\arraybackslash}X}
\newcolumntype{R}{>{\raggedleft\arraybackslash}X}

\begin{document}
\title{DDCA Project Final Report}
\maketitle
\tableofcontents

\pagebreak

\section{Objective}
The objective of this report is to showcase results from the  
Demand-Driven Cycamore Archetypes project (NEUP-FY16-10512). 

\section{Eg01-Eg23}

Figure \ref{fig:23flow} shows the flow of Eg01-Eg23.

\begin{figure}[H]
	\centering
	\includegraphics[width=\textwidth]{23-figures/23flow.pdf} 
	\hfill
	\caption{EG01-EG23.}
	\label{fig:23flow}
\end{figure}

\subsection{Flat Power Demand}

This section presents plots of power for all the prediction methods. The power demand is 60000 MW throughout the whole simulation. Table \ref{tab:23-inputs} shows the input file values. Figures \ref{fig:23-NO}, \ref{fig:23-DO}, and \ref{fig:23-SO} display the power demand supply for the Non-optimizing (NO), Deterministic-optimizing (DO), and Stochastic-optimizing methods (SO), respectively. The plots show the curves close to when the transition begins. It takes a small number of time steps in the beginning of the simulation until the supply meets the demand.
Table \ref{tab:23-power} records the number of steps with under supply, the cumulative under supply, and the cumulative oversupply. Cumulative under supply and cumulative oversupply represent the summation of the difference between the power supplied and the power demanded for all the time steps in the simulation. This magnitude could be best understood as energy. The cumulative under supply represents the energy not provided during the time steps in which the supply did not meet the demand. Likewise, the oversupply is the excess of energy produced.
In table \ref{tab:23-power} we see that the smallest cumulative under supply and smallest amount of under supply time steps are for poly and fft.

\begin{table}[H]
	\centering
	\caption{EG01-EG23 input file values.}
	\label{tab:23-inputs}
	\begin{tabularx}{\textwidth}{lR}
		\hline
		Parameter			& Value \\ 	\hline
		Demand equation		& 60e3  \\
		Installed Capacity 	& 1 \\
		Buffer    			& 0 \\
		Forward Steps		& 1 \\
		Backward Steps		& 2 \\		\hline
	\end{tabularx}
\end{table}

\begin{figure}[H]
	\centering
	\includegraphics[width=\textwidth]{23-figures/23-power0-buffer01.png} 
	\hfill
	\caption{Constant power demand of 60GW and power supply obtained with the NO algorithms.}
	\label{fig:23-NO}
\end{figure}

\begin{figure}[H]
	\centering
	\includegraphics[width=\textwidth]{23-figures/23-power0-buffer02.png} 
	\hfill
	\caption{Constant power demand of 60GW and power supply obtained with the DO algorithms.}
	\label{fig:23-DO}
\end{figure}

\begin{figure}[H]
	\centering
	\includegraphics[width=\textwidth]{23-figures/23-power0-buffer03.png} 
	\hfill
	\caption{Constant power demand of 60GW and power supply obtained with the SO algorithms.}
	\label{fig:23-SO}
\end{figure}

\begin{table}[H]
	\centering
	\caption{Under supply and oversupply of Power for the different prediction algorithms used to calculate EG01-EG23.}
	\label{tab:23-power}
	\begin{tabularx}{\textwidth}{lRRR}
		\hline
		Algorithm & Undersupplied & Cumulative  & Cumulative \\
		& Timesteps     & Undersupply [GW.mo]  & Oversupply [GW.mo] \\ \hline
		MA        & 26 	& 306.0 &  907.8   \\ 
		ARMA      & 26 	& 306.0 &  907.8   \\ 
		ARCH      & 26 	& 306.0 &  907.8   \\ 
		POLY      &  6 	& 235.0 &  2820.5  \\ 
		EXP\_SMOOTHING 	& 27 & 366.0 & 907.8 \\ 
		HOLT-WINTERS  	& 27 & 366.0 & 907.8 \\ 
		FFT       & 8	& 307.0	& 2820.5 \\ 
		SW\_SEASONAL    & 36 & 308.0 & 398.1	\\ \hline
	\end{tabularx}
\end{table}

\subsection{Buffer}

This section presents a sensitivity analysis for different values of the buffer. Figure \ref{fig:23-buff} shows a comparison of the cumulative under supply for different buffer sizes for different prediction methods. The cumulative under supply, remains constant for some of the methods and decreases with the increase of the buffer for others. Figure \ref{fig:23-buf-poly} displays the power demand and supply for different values of the buffer using poly. For this last case, the under supply remains constant. Figure \ref{fig:23-buf-poly} helps to understand the observed behavior. During at the transition we can see that even for the buffer with size 0 MW there is no under supply. Then, increasing the buffer will not decrease an under supply that is already zero.

\begin{figure}[H]
	\centering
	\includegraphics[width=\textwidth]{23-figures/23-sens-buffer.png} 
	\hfill
	\caption{Sensitivity analysis for different buffer sizes for different prediction algorithms.}
	\label{fig:23-buff}
\end{figure}

\begin{figure}[H]
	\centering
	\includegraphics[width=\textwidth]{23-figures/23-power-buffer-poly.png} 
	\hfill
	\caption{Power supply for different buffer sizes using poly.}
	\label{fig:23-buf-poly}
\end{figure}

\subsection{Forward Steps}

This section presents a sensitivity analysis for different values of forward steps chosen for the input files.
Figure \ref{fig:23-steps} shows a plot of the cumulative under supply for different values of forward steps using poly.
Slightly increasing the number of forward steps decreases the under supply. Increasing the number of forward steps too much has a negative impact on the results.
Figures \ref{fig:23-ste-poly} displays the plots for power supply for different number of forward steps. For 4 and 5 forward steps, the under supply increases.
Increasing the number of forward steps enlarges the production of power, more reactors are deployed and consequently, the new reactors require more fuel. As they require more fuel, the available fuel will not be enough, and the scenario will fail. Figure \ref{fig:23-ste-fft-mixerout} helps to understand this behavior. This figure shows that the demand of fuel for the FRs is larger than the supply of the same commodity.
The forward steps capability should be used only with small number of forward steps to avoid this under supply from happening.

\begin{figure}[H]
	\centering
	\includegraphics[width=\textwidth]{23-figures/23-sens-steps.png} 
	\hfill
	\caption{Cumulative under supply varying the number of forward steps using poly.}
	\label{fig:23-steps}
\end{figure}

\begin{figure}[H]
	\centering
	\includegraphics[width=\textwidth]{23-figures/23-power-buffer0-poly-steps.png} 
	\hfill
	\caption{Power supply varying the number of forward steps using poly.}
	\label{fig:23-ste-poly}
\end{figure}

\begin{figure}[H]
	\centering
	\includegraphics[width=\textwidth]{23-figures/0-S5-poly-mixerout.png} 
	\hfill
	\caption{Power supply for 5 forward steps using poly.}
	\label{fig:23-ste-fft-mixerout}
\end{figure}

\subsection{Different commodities}

Since the prediction algorithm poly performed the best, this section presents plots for the supply and demand of the most meaningful commodities in the scenario.
Table \ref{tab:23-commodities} summarizes the figures in this section and the respective commodities.
Table \ref{tab:23-commod} presents the number of steps of under supply, cumulative under supply, and cumulative oversupply of such commodities.

\begin{table}[H]
	\centering
	\caption{Commodity names used in the simulation of EG01-EG23.}
	\label{tab:23-commodities}
	\begin{tabularx}{\textwidth}{lL}
		\hline
		Commodity & Figure \\ \hline
  		Power           & \ref{fig:23-power} \\
		Natural-U       & \ref{fig:23-sourceout} \\
        Enriched-U   	& \ref{fig:23-enrichmentout} \\
        FR fuel       	& \ref{fig:23-mixerout} \\
  		Reprocessed Pu from spent fuel of LWRs & \ref{fig:23-lwrpu} \\
  		Reprocessed Pu from spent fuel of FRs  & \ref{fig:23-frpu} \\ \hline
	\end{tabularx}
\end{table}

\begin{figure}[H]
	\centering
	\includegraphics[width=\textwidth]{23-figures/0-poly-power.png} 
	\hfill
	\caption{Demand and supply of Power and number of reactors deployed.}
	\label{fig:23-power}
\end{figure}

\begin{figure}[H]
	\centering
	\begin{subfigure}[]{0.45\textwidth}
		\centering
		\includegraphics[width=\linewidth]{23-figures/0-poly-sourceout.png} 
		\caption{Natural-U.}
		\label{fig:23-sourceout}
	\end{subfigure}
	\vspace{1cm}
	\begin{subfigure}[]{0.45\textwidth}
		\centering
		\includegraphics[width=\linewidth]{23-figures/0-poly-enrichmentout.png} 
		\caption{Enriched-U.}
		\label{fig:23-enrichmentout}
	\end{subfigure}
	\hfill
	\caption{Demand and supply of different commodities and number of facilities that produce them.}
	\label{fig:23-front}
\end{figure}

\begin{figure}[H]
	\centering
	\includegraphics[width=\textwidth]{23-figures/0-poly-mixerout.png} 
	\hfill
	\caption{Demand and supply of FR fuel and number of FR reactors.}
	\label{fig:23-mixerout}
\end{figure}

\begin{figure}[H]
	\centering
	\begin{subfigure}[t]{0.45\textwidth}
		\centering
		\includegraphics[width=\linewidth]{23-figures/0-poly-lwrpu.png} 
		\caption{Reprocessed Pu from spent fuel of LWRs.}
		\label{fig:23-lwrpu}
	\end{subfigure}
	\vspace{1cm}
	\begin{subfigure}[t]{0.45\textwidth}
		\centering
		\includegraphics[width=\linewidth]{23-figures/0-poly-frpu.png} 
		\caption{Reprocessed Pu from spent fuel of FRs.}
		\label{fig:23-frpu}
	\end{subfigure}
	\hfill
	\caption{Demand and supply of different commodities and number of facilities supplied with them.}
	\label{fig:23-pu}
\end{figure}

\begin{table}[H]
	\centering
	\caption{Under supply and oversupply of different commodities using poly to calculate EG01-EG23.}
	\label{tab:23-commod}
	\begin{tabularx}{\textwidth}{lRRR}
		\hline
		Commodity & Undersupplied & Cumulative  & Cumulative \\
		& Timesteps     & Undersupply [$10^3$Kg]  & Oversupply [$10^6$Kg] \\ \hline
		Natural-U & 2 & 4713  &  35648   \\ 
		Enriched-U & 4 & 48.5$.10^3$  &  42259   \\ 
        Reprocessed Pu from spent fuel of LWRs & 1 & 1.7 & - \\
        Reprocessed Pu from spent fuel of FRs & 1 & 27.4 & - \\ \hline
	\end{tabularx}
\end{table}

\section{Eg01-Eg24}

Figure \ref{fig:24flow} shows the flow of Eg01-Eg24.

\begin{figure}[H]
	\centering
	\includegraphics[width=\textwidth]{24-figures/24flow.pdf} 
	\hfill
	\caption{EG01-EG24.}
	\label{fig:24flow}
\end{figure}

\subsection{Linearly increasing Power Demand}

This section presents plots of power for all the prediction methods. The power demand increases linearly with the expression $60000 MW + 250*t MW/year$. Table \ref{tab:24-inputs} shows the input file values. Figures \ref{fig:24-lin-NO}, \ref{fig:24-lin-DO}, and \ref{fig:24-lin-SO} display the power demand supply for the Non-optimizing (NO), Deterministic-optimizing (DO), and Stochastic-optimizing methods (SO), respectively. The plots show the curves close to when the transition begins. It takes a small number of time steps in the beginning of the simulation until the supply meets the demand.
Table \ref{tab:24-lin-power} records the number of steps whit under supply, the cumulative under supply, and the cumulative oversupply. The smallest cumulative under supply and smallest amount of under supply time steps are for fft.

\begin{table}[H]
	\centering
	\caption{EG01-EG24 input file values.}
	\label{tab:24-inputs}
	\begin{tabularx}{\textwidth}{lR}
		\hline
		Parameter			& Value \\ 	\hline
		Demand equation		& 60GW + 250MW/year*t  \\
		Installed Capacity 	& 1 \\
		Buffer    			& 0 \\
		Forward Steps		& 1 \\
		Backward Steps		& 2 \\		\hline
	\end{tabularx}
\end{table}

\begin{figure}[H]
	\centering
	\includegraphics[width=\textwidth]{24-figures/lin-24-power-buffer01.png} 
	\hfill
	\caption{Linearly increasing power demand of 250MW/y and power supply obtained with the NO algorithms.}
	\label{fig:24-lin-NO}
\end{figure}

\begin{figure}[H]
	\centering
	\includegraphics[width=\textwidth]{24-figures/lin-24-power-buffer02.png} 
	\hfill
	\caption{Linearly increasing power demand of 250MW/y and power supply obtained with the DO algorithms.}
	\label{fig:24-lin-DO}
\end{figure}

\begin{figure}[H]
	\centering
	\includegraphics[width=\textwidth]{24-figures/lin-24-power-buffer03.png} 
	\hfill
	\caption{Linearly increasing power demand of 250MW/y and power supply obtained with the SO algorithms.}
	\label{fig:24-lin-SO}
\end{figure}

\begin{table}[H]
	\centering
	\caption{Under supply and oversupply of Power for the different prediction algorithms used to calculate EG01-EG24.}
	\label{tab:24-lin-power}
	\begin{tabularx}{\textwidth}{lRRR}
		\hline
		Algorithm & Undersupplied & Cumulative  & Cumulative \\
		& Timesteps     & Undersupply [GW.mo]  & Oversupply [GW.mo] \\ \hline
		MA        & 36 	& 313.7 & 840.9 \\ 
		ARMA      & 36 	& 313.7 & 840.9 \\ 
		ARCH      & 36 	& 316.8 & 859.0 \\ 
		POLY      &  65 & 282.4 & 1974.7 \\ 
		EXP\_SMOOTHING 	& 37 & 373.4 & 828.7 \\ 
		HOLT-WINTERS  	& 37 & 373.4 & 828.7 \\ 
		FFT       & 20	& 315.1	& 2019.1 \\ 
		SW\_SEASONAL    & 107 & 318.8 & 579.09 \\ \hline
	\end{tabularx}
\end{table}

\subsection{Buffer}

This section presents a sensitivity analysis for different values of the buffer. Figure \ref{fig:24-buff} shows a comparison of the cumulative under supply for different buffer sizes using fft. Figure \ref{fig:24-buf-fft} displays the power demand and supply for different values of the buffer using fft. The cumulative under supply decreases with the increase of the buffer, reaching an asymptotic value. That value is given by the initialization of the scenario, when the buffer does not affect considerably the under supply.

\begin{figure}[H]
	\centering
	\includegraphics[width=\textwidth]{24-figures/24-sens-buffer.png} 
	\hfill
	\caption{Sensitivity analysis for different buffer sizes using fft.}
	\label{fig:24-buff}
\end{figure}

\begin{figure}[H]
	\centering
	\includegraphics[width=\textwidth]{24-figures/24-power-buffer-fft.png} 
	\hfill
	\caption{Power supply for different buffer sizes using fft.}
	\label{fig:24-buf-fft}
\end{figure}

\section{Eg01-Eg29}

Figure \ref{fig:29flow} shows the flow of Eg01-Eg29.

\begin{figure}[H]
	\centering
	\includegraphics[width=\textwidth]{29-figures/29flow.pdf} 
	\hfill
	\caption{EG01-EG29.}
	\label{fig:29flow}
\end{figure}

\subsection{Flat Power Demand}

This section presents plots of power for all the prediction methods. The power demand is 60000 MW throughout the whole simulation. Table \ref{tab:29-inputs} shows the input file values. Figures \ref{fig:29-NO}, \ref{fig:29-DO}, and \ref{fig:29-SO} display the power demand supply for the Non-optimizing (NO), Deterministic-optimizing (DO), and Stochastic-optimizing methods (SO), respectively.
Table \ref{tab:29-power} records the number of steps with under supply, the cumulative under supply, and the cumulative oversupply. The smallest cumulative under supply and smallest amount of under supply time steps are for poly and fft.

\begin{table}[H]
	\centering
	\caption{EG01-EG29 input file values.}
	\label{tab:29-inputs}
	\begin{tabularx}{\textwidth}{lR}
		\hline
		Parameter			& Value \\ 	\hline
		Demand equation		& 60e3  \\
		Installed Capacity 	& 1 \\
		Buffer    			& 0 \\
		Forward Steps		& 1 \\
		Backward Steps		& 2 \\		\hline
	\end{tabularx}
\end{table}

\begin{figure}[H]
	\centering
	\includegraphics[width=\textwidth]{29-figures/29-power0-buffer01.png} 
	\hfill
	\caption{Constant power demand of 60GW and power supply obtained with the SO algorithms.}
	\label{fig:29-NO}
\end{figure}

\begin{figure}[H]
	\centering
	\includegraphics[width=\textwidth]{29-figures/29-power0-buffer02.png} 
	\hfill
	\caption{Constant power demand of 60GW and power supply obtained with the DO algorithms.}
	\label{fig:29-DO}
\end{figure}

\begin{figure}[H]
	\centering
	\includegraphics[width=\textwidth]{29-figures/29-power0-buffer03.png} 
	\hfill
	\caption{Constant power demand of 60GW and power supply obtained with the SO algorithms.}
	\label{fig:29-SO}
\end{figure}

\begin{table}[H]
	\centering
	\caption{Under supply and oversupply of Power for the different prediction algorithms used to calculate EG01-EG29.}
	\label{tab:29-power}
	\begin{tabularx}{\textwidth}{lRRR}
		\hline
		Algorithm & Undersupplied & Cumulative  & Cumulative \\
		& Timesteps     & Undersupply [GW.mo]  & Oversupply [GW.mo] \\ \hline
		MA        & 15 	& 145.0 & 1847.0 \\ 
		ARMA      & 15 	& 145.0 & 1847.0 \\ 
		ARCH      & 15 	& 145.0 & 1846.9 \\ 
		POLY      &  4 	& 90.0 & 4720.3 \\ 
		EXP\_SMOOTHING 	& 16 & 205.0 & 1847.0 \\ 
		HOLT-WINTERS  	& 16 & 205.0 & 1847.0 \\ 
		FFT       &  5	& 150.0	& 4898.0 \\ 
		SW\_SEASONAL    & 14 & 139.0 & 798.9 \\ \hline
	\end{tabularx}
\end{table}

\subsection{Different commodities}

Since the prediction algorithm poly performed the best, this section presents plots for the supply and demand of the most meaningful commodities in the scenario.
Table \ref{tab:29-commodities} summarizes the figures in this section and the respective commodities.
Table \ref{tab:29-commod} presents the number of steps of under supply, cumulative under supply, and cumulative oversupply of such commodities.

\begin{table}[H]
	\centering
	\caption{Commodity names used in the simulation of EG01-EG29.}
	\label{tab:29-commodities}
	\begin{tabularx}{\textwidth}{lL}
		\hline
		Commodity & Figure \\ \hline
		Power           & \ref{fig:29-power} \\
		Natural-U       & \ref{fig:29-sourceout} \\
		Enriched-U   	& \ref{fig:29-enrichmentout} \\
		FR fuel       	& \ref{fig:29-frmixerout} \\
		MOX LWR fuel   	& \ref{fig:29-moxmixerout} \\
		Reprocessed Pu from spent fuel of LWRs & \ref{fig:29-pu1} \\
		Reprocessed Pu from spent fuel of FRs  & \ref{fig:29-frpu} \\
		Reprocessed Pu from spent fuel of MOX LWRs  & \ref{fig:29-moxpu} \\ \hline
	\end{tabularx}
\end{table}

\begin{figure}[H]
	\centering
	\includegraphics[width=\textwidth]{29-figures/0-poly-power.png} 
	\hfill
	\caption{Demand and supply of Power and number of reactors deployed.}
	\label{fig:29-power}
\end{figure}

\begin{figure}[H]
	\centering
	\begin{subfigure}[]{0.45\textwidth}
		\centering
		\includegraphics[width=\linewidth]{29-figures/0-poly-sourceout.png} 
		\caption{Natural-U.}
		\label{fig:29-sourceout}
	\end{subfigure}
	\vspace{1cm}
	\begin{subfigure}[]{0.45\textwidth}
		\centering
		\includegraphics[width=\linewidth]{29-figures/0-poly-enrichmentout.png} 
		\caption{Enriched-U.}
		\label{fig:29-enrichmentout}
	\end{subfigure}
	\hfill
	\caption{Demand and supply of different commodities and number of facilities that produce them.}
	\label{fig:29-front}
\end{figure}

\begin{figure}[H]
	\centering
	\begin{subfigure}[]{0.45\textwidth}
		\centering
		\includegraphics[width=\linewidth]{29-figures/0-poly-frmixerout.png} 
		\caption{FR Fuel.}
		\label{fig:29-frmixerout}
	\end{subfigure}
	\vspace{1cm}
	\begin{subfigure}[]{0.45\textwidth}
		\centering
		\includegraphics[width=\linewidth]{29-figures/0-poly-moxmixerout.png} 
		\caption{MOX LWR Fuel.}
		\label{fig:29-moxmixerout}
	\end{subfigure}
	\hfill
	\caption{Demand and supply of fuel and number of reactors.}
	\label{fig:29-mix}
\end{figure}

\begin{figure}[H]
	\centering
	\includegraphics[width=\textwidth]{29-figures/0-poly-lwrpu.png} 
	\hfill
	\caption{Demand and supply of reprocessed Pu from spent fuel of LWRs and number of facilities supplied with them.}
	\label{fig:29-pu1}
\end{figure}

\begin{figure}[H]
	\centering
	\begin{subfigure}[t]{0.45\textwidth}
		\centering
		\includegraphics[width=\linewidth]{23-figures/0-poly-frpu.png} 
		\caption{Reprocessed Pu from spent fuel of FRs.}
		\label{fig:29-frpu}
	\end{subfigure}
	\vspace{1cm}
	\begin{subfigure}[t]{0.45\textwidth}
		\centering
		\includegraphics[width=\linewidth]{29-figures/0-poly-moxpu.png} 
		\caption{Commodity moxpu.}
		\label{fig:29-moxpu}
	\end{subfigure}
	\hfill
	\caption{Demand and supply of different commodities and number of facilities supplied with them.}
	\label{fig:29-pu2}
\end{figure}

\begin{table}[H]
	\centering
	\caption{Under supply and oversupply of different commodities using poly to calculate EG01-EG29.}
	\label{tab:29-commod}
	\begin{tabularx}{\textwidth}{lRRR}
		\hline
		Commodity & Undersupplied & Cumulative  & Cumulative \\
		& Timesteps & Undersupply [$10^3$Kg]  & Oversupply [$10^6$Kg] \\ \hline
		Natural-U & 1 & 34394.0  & 132319869.5 \\ 
		Enriched-U & 1 & 16126.1 & 102751545581.6 \\
		FR Fuel & 2 & 284.4 & 124827.3 \\
        MOX LWR Fuel & 2 & 530.1 & 354541.5 \\ \hline
	\end{tabularx}
\end{table}

\section{Eg01-Eg30}

Figure \ref{fig:30flow} shows the flow of Eg01-Eg30.

\begin{figure}[H]
	\centering
	\includegraphics[width=\textwidth]{30-figures/30flow.pdf} 
	\hfill
	\caption{EG01-EG30.}
	\label{fig:30flow}
\end{figure}

\subsection{Linearly increasing Power Demand}

This section presents plots of power for all the prediction methods. The power demand increases linearly with the expression $60000 MW + 250*t MW/year$. The input files use the installed capacity feature. Buffer is set to zero, back steps is set to two, and steps takes the default value of one.
Figures \ref{fig:30-lin-NO}, \ref{fig:30-lin-DO}, and \ref{fig:30-lin-SO} display the power supply and demand. The plots show the curves close to when the transition begins. It takes a small number of time steps in the beginning of the simulation until the supply meets the demand.
Table \ref{tab:30-lin-power} records the number of steps whit under supply, the cumulative under supply, and the cumulative oversupply. The smallest cumulative under supply and smallest amount of under supply time steps are for poly and fft.

\begin{table}[H]
	\centering
	\caption{EG01-EG30 input file values.}
	\label{tab:30-inputs}
	\begin{tabularx}{\textwidth}{lR}
		\hline
		Parameter			& Value \\ 	\hline
		Demand equation		& 60GW + 250MW/year*t  \\
		Installed Capacity 	& 1 \\
		Buffer    			& 0 \\
		Forward Steps		& 1 \\
		Backward Steps		& 2 \\		\hline
	\end{tabularx}
\end{table}

\begin{figure}[H]
	\centering
	\includegraphics[width=\textwidth]{30-figures/lin-30-power-buffer01.png} 
	\hfill
	\caption{Linearly increasing power demand of 250MW/y and power supply obtained with the NO algorithms.}
	\label{fig:30-lin-NO}
\end{figure}

\begin{figure}[H]
	\centering
	\includegraphics[width=\textwidth]{30-figures/lin-30-power-buffer02.png} 
	\hfill
	\caption{Linearly increasing power demand of 250MW/y and power supply obtained with the DO algorithms.}
	\label{fig:30-lin-DO}
\end{figure}

\begin{figure}[H]
	\centering
	\includegraphics[width=\textwidth]{30-figures/lin-30-power-buffer03.png} 
	\hfill
	\caption{Linearly increasing power demand of 250MW/y and power supply obtained with the SO algorithms.}
	\label{fig:30-lin-SO}
\end{figure}

\begin{table}[H]
	\centering
	\caption{Under supply and oversupply of Power for the different prediction algorithms used to calculate EG01-EG24.}
	\label{tab:30-lin-power}
	\begin{tabularx}{\textwidth}{lRRR}
		\hline
		Algorithm & Undersupplied & Cumulative  & Cumulative \\
		& Timesteps     & Undersupply [GW.mo]  & Oversupply [GW.mo] \\ \hline
		MA        & 24 & 152.3 & 1334.1 \\ 
		ARMA      & 24 & 152.3 & 1334.1 \\ 
		ARCH      & 21 & 152.1 & 1355.9 \\ 
		POLY      &  9 & 92.5 & 3073.1 \\ 
		EXP\_SMOOTHING 	& 25 & 211.6 & 1317.8 \\ 
		HOLT-WINTERS  	& 25 & 211.6 & 1317.8 \\ 
		FFT       & 9 & 152.5 & 3079.4 \\ 
		SW\_SEASONAL  & 51 & 147.3 & 873.4 \\ \hline
	\end{tabularx}
\end{table}

\end{document}