\documentclass[11pt]{article}
\usepackage[utf8]{inputenc} % Required for inputting international characters
\usepackage[T1]{fontenc} % Output font encoding for international characters
\usepackage{caption} % for table captions
\usepackage{amsmath} % for multi-line equations and piecewises
\DeclareMathOperator{\sign}{sign}
\usepackage{graphicx}
\usepackage{relsize}
\usepackage{xspace}
\usepackage{verbatim} % for block comments
\usepackage{subcaption} % for subfigures
\usepackage{enumitem} % for a) b) c) lists
\newcommand{\Cyclus}{\textsc{Cyclus}\xspace}%
\newcommand{\Cycamore}{\textsc{Cycamore}\xspace}%
\newcommand{\deploy}{\texttt{d3ploy}\xspace}%
\newcommand{\Deploy}{\texttt{D3ploy}\xspace}%
\usepackage{tabularx}
\usepackage{color}
\usepackage{multirow}
\usepackage[acronym,toc]{glossaries}
\include{acros}
\definecolor{bg}{rgb}{0.95,0.95,0.95}
\newcolumntype{b}{X}
\newcolumntype{f}{>{\hsize=.15\hsize}X}
\newcolumntype{s}{>{\hsize=.5\hsize}X}
\newcolumntype{m}{>{\hsize=.75\hsize}X}
\newcolumntype{r}{>{\hsize=1.1\hsize}X}
\usepackage{titling}
\usepackage[hang,flushmargin]{footmisc}
\renewcommand*\footnoterule{}
\usepackage{tikz}
\definecolor{illiniblue}{HTML}{437db6}
\definecolor{illiniorange}{HTML}{E38749}
\usetikzlibrary{shapes.geometric, arrows}
\tikzstyle{oblock} = [rectangle, draw, fill=illiniorange, 
text width=12em, text centered, rounded corners, minimum height=4em]
\tikzstyle{bblock} = [rectangle, draw, fill=illiniblue, 
text width=12em, text centered, rounded corners, minimum height=4em]
\tikzstyle{arrow} = [thick,->,>=stealth]

\usetikzlibrary{shapes.geometric,arrows}
\tikzstyle{process} = [rectangle, rounded corners, 
minimum width=1cm, minimum height=1cm,text centered, draw=black, 
fill=blue!30]
\tikzstyle{arrow} = [thick,->,>=stealth]

\graphicspath{}
\usepackage{mathpazo} % Palatino font
\usepackage{graphicx} % For the logo
\usepackage{float} 

\usepackage[hidelinks]{hyperref}
\urlstyle{tt}

\newcolumntype{L}{>{\raggedright\arraybackslash}X}
\newcolumntype{R}{>{\raggedleft\arraybackslash}X}

\begin{document}
\title{DDCA Project Final Report}
\maketitle
\tableofcontents

\pagebreak

\section{Objective}
The objective of this report is to showcase results from the  
Demand-Driven Cycamore Archetypes project (NEUP-FY16-10512). 

\section{Eg01-Eg23}

Figure \ref{fig:23flow} shows the flow of Eg01-Eg23.

\begin{figure}[H]
	\centering
	\includegraphics[width=\textwidth]{23-figures/23flow.pdf} 
	\hfill
	\caption{EG01-EG23.}
	\label{fig:23flow}
\end{figure}

\subsection{Power}

This section presents plots of power for all the prediction methods. The power demand is 60000 MW throughout the whole simulation. The input files use the installed capacity feature. Buffer is set to zero, back steps is set to two, and steps takes the default value of one.
Figures \ref{fig:23-NO}, \ref{fig:23-DO}, and \ref{fig:23-SO} display the power supply and demand.
Table \ref{tab:23-power} records the number of steps whit under supply, the cumulative under supply, and the cumulative oversupply.

\begin{figure}[H]
	\centering
	\includegraphics[width=\textwidth]{23-figures/23-power-buffer01.png} 
	\hfill
	\caption{NO algorithms.}
	\label{fig:23-NO}
\end{figure}

\begin{figure}[H]
	\centering
	\includegraphics[width=\textwidth]{23-figures/23-power-buffer02.png} 
	\hfill
	\caption{DO algorithms.}
	\label{fig:23-DO}
\end{figure}

\begin{figure}[H]
	\centering
	\includegraphics[width=\textwidth]{23-figures/23-power-buffer03.png} 
	\hfill
	\caption{SO algorithms.}
	\label{fig:23-SO}
\end{figure}

\begin{table}[H]
	\centering
	\caption{Undersupply and oversupply of Power for the different algorithms used to calculate EG01-EG23.}
	\label{tab:23-power}
	\begin{tabularx}{\textwidth}{lRRR}
		\hline
		Algorithm & Undersupplied & Cumulative  & Cumulative \\
		& Timesteps     & Undersupply [GW.mo]  & Oversupply [GW.mo] \\ \hline
		MA        & 26 	& 306.0 &  907.8   \\ 
		ARMA      & 26 	& 306.0 &  907.8   \\ 
		ARCH      & 26 	& 306.0 &  907.8   \\ 
		POLY      &  6 	& 235.0 &  2820.5  \\ 
		EXP\_SMOOTHING 	& 27 & 366.0 & 907.8 \\ 
		HOLT-WINTERS  	& 27 & 366.0 & 907.8 \\ 
		FFT       & 8	& 307.0	& 2820.5 \\ 
		SW\_SEASONAL    & 36 & 308.0 & 398.1	\\ \hline
	\end{tabularx}
\end{table}

\subsection{Buffer}

This section presents a sensitivity analysis for different values of the buffer. Figure \ref{fig:23-buff} shows a comparison of the cumulative under supply for different buffer sizes.

Figures \ref{fig:23-buf-ma} to \ref{fig:23-buf-fft} display a comparison for some of the methods of the power supply for different buffer sizes.

The input files use the installed capacity feature. Buffer takes the values 0, 2000, 4000, 6000, and 8000. Back steps is set to two, and steps takes the default value of one.

\begin{figure}[H]
	\centering
	\includegraphics[width=\textwidth]{23-figures/23-sens-buffer.png} 
	\hfill
	\caption{Sensitivity analysis for different buffer sizes for some prediction algorithms.}
	\label{fig:23-buff}
\end{figure}

\begin{figure}[H]
	\centering
	\includegraphics[width=\textwidth]{23-figures/23-power-buffer-ma.png} 
	\hfill
	\caption{Power supply for different buffer sizes using ma.}
	\label{fig:23-buf-ma}
\end{figure}

\begin{figure}[H]
	\centering
	\includegraphics[width=\textwidth]{23-figures/23-power-buffer-arma.png} 
	\hfill
    \caption{Power supply for different buffer sizes using arma.}
	\label{fig:23-buf-arma}
\end{figure}

\begin{figure}[H]
	\centering
	\includegraphics[width=\textwidth]{23-figures/23-power-buffer-arch.png} 
	\hfill
    \caption{Power supply for different buffer sizes using arch.}
	\label{fig:23-buf-arch}
\end{figure}

\begin{figure}[H]
	\centering
	\includegraphics[width=\textwidth]{23-figures/23-power-buffer-poly.png} 
	\hfill
    \caption{Power supply for different buffer sizes using poly.}
	\label{fig:23-buf-poly}
\end{figure}

\begin{figure}[H]
	\centering
	\includegraphics[width=\textwidth]{23-figures/23-power-buffer-exp_smoothing.png} 
	\hfill
    \caption{Power supply for different buffer sizes using exp\_smoothing.}
	\label{fig:23-buf-exp_smoothing}
\end{figure}

\begin{figure}[H]
	\centering
	\includegraphics[width=\textwidth]{23-figures/23-power-buffer-holt_winters.png} 
	\hfill
	\caption{Power supply for different buffer sizes using holt\_winters.}
	\label{fig:23-buf-hots_winters}
\end{figure}

\begin{figure}[H]
	\centering
	\includegraphics[width=\textwidth]{23-figures/23-power-buffer-fft.png} 
	\hfill
    \caption{Power supply for different buffer sizes using fft.}
	\label{fig:23-buf-fft}
\end{figure}

\subsection{Steps forward}

This section presents a sensitivity analysis for different values of steps.
Figure \ref{fig:23-steps} shows a comparison of the cumulative under supply for different values of steps forward.

Figures \ref{fig:23-ste-ma} to \ref{fig:23-ste-fft} display a comparison for some of the methods of the power supply for different steps.

The input files use the installed capacity feature. Buffer is set to zero, back steps is set to two, and steps takes the values 1, 2, 3, 4, and 5.

Seeing figure \ref{fig:23-steps} we note that the more steps used, the worse the simulation performs. Figure \ref{fig:23-ste-fft-mixerout} helps to understand such behavior. 'Mixerout' is the commodity that represents the fuel that FR use. This figure shows that the LWRs produce enough fuel to almost start all the FRs. However, this expense of fuel is too big to keep the ones already deployed running until they produce their own fuel, so their power supply oscillates. The steps capability should be used cautiously to avoid this from happening.

\begin{figure}[H]
	\centering
	\includegraphics[width=\textwidth]{23-figures/23-sens-steps.png} 
	\hfill
	\caption{Sensitivity analysis for different number of steps forward for some prediction algorithms.}
	\label{fig:23-steps}
\end{figure}

\begin{figure}[H]
	\centering
	\includegraphics[width=\textwidth]{23-figures/23-power-buffer0-ma-steps.png} 
	\hfill
	\caption{Power supply for different values of steps forward using ma.}
	\label{fig:23-ste-ma}
\end{figure}

\begin{figure}[H]
	\centering
	\includegraphics[width=\textwidth]{23-figures/23-power-buffer0-arma-steps.png} 
	\hfill
	\caption{Power supply for different values of steps forward using arma.}
	\label{fig:23-ste-arma}
\end{figure}

\begin{figure}[H]
	\centering
	\includegraphics[width=\textwidth]{23-figures/23-power-buffer0-arch-steps.png} 
	\hfill
	\caption{Power supply for different values of steps forward using arch.}
	\label{fig:23-ste-arch}
\end{figure}

\begin{figure}[H]
	\centering
	\includegraphics[width=\textwidth]{23-figures/23-power-buffer0-poly-steps.png} 
	\hfill
	\caption{Power supply for different values of steps forward using poly.}
	\label{fig:23-ste-poly}
\end{figure}

\begin{figure}[H]
	\centering
	\includegraphics[width=\textwidth]{23-figures/23-power-buffer0-exp_smoothing-steps.png} 
	\hfill
	\caption{Power supply for different values of steps forward using exp\_smoothing.}
	\label{fig:23-ste-exp_smoothing}
\end{figure}

\begin{figure}[H]
	\centering
	\includegraphics[width=\textwidth]{23-figures/23-power-buffer0-holt_winters-steps.png} 
	\hfill
	\caption{Power supply for different values of steps forward using holt\_winters.}
	\label{fig:23-ste-hots_winters}
\end{figure}

\begin{figure}[H]
	\centering
	\includegraphics[width=\textwidth]{23-figures/23-power-buffer0-fft-steps.png} 
	\hfill
	\caption{Power supply for different values of steps forward using fft.}
	\label{fig:23-ste-fft}
\end{figure}

\begin{figure}[H]
	\centering
	\includegraphics[width=\textwidth]{23-figures/0-S4-fft-mixerout.png} 
	\hfill
	\caption{Power supply for 4 steps forward using fft.}
	\label{fig:23-ste-fft-mixerout}
\end{figure}

\subsection{Back steps}

This section presents a sensitivity analysis for different values of  back steps. Figure \ref{fig:23-backs} shows a comparison of the cumulative under supply for different values of back steps.

Figures \ref{fig:23-back-ma} to \ref{fig:23-back-fft} display a comparison for some of the methods of the power supply for different values of back steps.

The input files use the installed capacity feature. The buffer is set to zero, steps is set to two, and back steps takes the values 1, 2, 3, 4, and 5.

\begin{figure}[H]
	\centering
	\includegraphics[width=\textwidth]{23-figures/23-sens-backs.png} 
	\hfill
	\caption{Sensitivity analysis for different number of back steps for some prediction algorithms.}
	\label{fig:23-backs}
\end{figure}

\begin{figure}[H]
	\centering
	\includegraphics[width=\textwidth]{23-figures/23-power-buffer0-ma-back.png} 
	\hfill
	\caption{Power supply for different values of back steps using ma.}
	\label{fig:23-back-ma}
\end{figure}

\begin{figure}[H]
	\centering
	\includegraphics[width=\textwidth]{23-figures/23-power-buffer0-arma-back.png} 
	\hfill
	\caption{Power supply for different values of back steps using arma.}
	\label{fig:23-back-arma}
\end{figure}

\begin{figure}[H]
	\centering
	\includegraphics[width=\textwidth]{23-figures/23-power-buffer0-arch-back.png} 
	\hfill
	\caption{Power supply for different values of back steps using arch.}
	\label{fig:23-back-arch}
\end{figure}

\begin{figure}[H]
	\centering
	\includegraphics[width=\textwidth]{23-figures/23-power-buffer0-poly-back.png} 
	\hfill
	\caption{Power supply for different values of back steps using poly.}
	\label{fig:23-back-poly}
\end{figure}

\begin{figure}[H]
	\centering
	\includegraphics[width=\textwidth]{23-figures/23-power-buffer0-exp_smoothing-back.png} 
	\hfill
	\caption{Power supply for different values of back steps using exp\_smoothing.}
	\label{fig:23-back-exp_smoothing}
\end{figure}

\begin{figure}[H]
	\centering
	\includegraphics[width=\textwidth]{23-figures/23-power-buffer0-holt_winters-back.png} 
	\hfill
	\caption{Power supply for different values of back steps using holt\_winters.}
	\label{fig:23-back-hots_winters}
\end{figure}

\begin{figure}[H]
	\centering
	\includegraphics[width=\textwidth]{23-figures/23-power-buffer0-fft-back.png} 
	\hfill
	\caption{Power supply for different values of back steps using fft.}
	\label{fig:23-back-fft}
\end{figure}

\subsection{Different commodities}

Table \ref{tab:23-commodities} shows the name of the variables used in the simulations and what they represent in the cycle. Table \ref{tab:23-commod} presents the number of steps of under supply, cumulative under supply, and cumulative oversupply for some of the commodities.

Table \ref{tab:23-commod} needs some further explanation. The simulation differentiates between front-end and back-end commodities. In this simulation, front-end commodities are power, sourceout, and enrichmentout. All the rest of the commodities are back-end. 

For the first ones, under supply means that a facility requires that commodity and the supply is not enough. Over supply, means the opposite. A facility requires that commodity but there is too much of it. In other words, the facilities that provide such commodity are over sized. D3ploy can avoid this situation by deploying more facilities with smaller sizes.

For the back-end commodities, the notion of under supply and over supply are different. This facilities are meant to have an accumulation of material, so they have to over sized. For example, a sink has to be large enough so we do not have to deploy new continuously. It could still happen that the sink gets full and there is a need to build a new one. This is measure by the under supply. When there is too much material that needs to be disposed and the current sinks cannot take that quantity, there is a time step with under supply. Consequently, d3ploy will deploy a new facility on the next time step.

\begin{table}[H]
	\centering
	\caption{Commodity names used in the simulation of EG01-EG23.}
	\label{tab:23-commodities}
	\begin{tabularx}{\textwidth}{lLL}
		\hline
		Commodity name  & Figure & Represents \\ \hline
  		power           & \ref{fig:23-power} & Power \\
		sourceout       & \ref{fig:23-sourceout} & Natural-U \\
        enrichmentout   & \ref{fig:23-enrichmentout} & Enriched-U \\
        mixerout        & \ref{fig:23-mixerout} & FR fuel \\
  		lwrout          & \ref{fig:23-lwrout} & Spent fuel of LWRs \\
  		frout           & \ref{fig:23-frout} & Spent fuel of FRs \\
  		lwrstorageout   & \ref{fig:23-lwrstorageout} & Cooled down spent fuel of LWRs \\
  		frstorageout    & \ref{fig:23-frstorageout} & Cooled down spent fuel of FRs \\	
  		lwrpu   & \ref{fig:23-lwrpu} & Pu from spent fuel of LWRs \\
  		frpu    & \ref{fig:23-frpu} & Pu from spent fuel of FRs \\
  		lwrreprocessingwaste & \ref{fig:23-lwrreprocessingwaste} & Waste from the reprocessing of spent fuel of LWRs \\
 		frreprocessingwaste & \ref{fig:23-frreprocessingwaste} & Waste from the reprocessing of spent fuel of FRs \\ \hline

	\end{tabularx}
\end{table}

\begin{figure}[H]
	\centering
	\includegraphics[width=\textwidth]{23-figures/0-poly-power.png} 
	\hfill
	\caption{Supply and demand of the commodity power.}
	\label{fig:23-power}
\end{figure}

\begin{figure}[H]
	\centering
	\begin{subfigure}[]{0.45\textwidth}
		\centering
		\includegraphics[width=\linewidth]{23-figures/0-poly-sourceout.png} 
		\caption{Commodity sourceout.}
		\label{fig:23-sourceout}
	\end{subfigure}
	\vspace{1cm}
	\begin{subfigure}[]{0.45\textwidth}
		\centering
		\includegraphics[width=\linewidth]{23-figures/0-poly-enrichmentout.png} 
		\caption{Commodity enrichmentout.}
		\label{fig:23-enrichmentout}
	\end{subfigure}
	\hfill
	\caption{Supply and demand of different commodities for the prediction method poly.}
	\label{fig:23-front}
\end{figure}

\begin{figure}[H]
	\centering
	\includegraphics[width=\textwidth]{23-figures/0-poly-mixerout.png} 
	\hfill
	\caption{Supply and demand of the commodity mixerout.}
	\label{fig:23-mixerout}
\end{figure}

\begin{figure}[H]
	\centering
	\begin{subfigure}[]{0.45\textwidth}
		\centering
		\includegraphics[width=\linewidth]{23-figures/0-poly-lwrout.png} 
		\caption{Commodity lwrout.}
		\label{fig:23-lwrout}
	\end{subfigure}
	\vspace{1cm}
	\begin{subfigure}[]{0.45\textwidth}
		\centering
		\includegraphics[width=\linewidth]{23-figures/0-poly-frout.png} 
		\caption{Commodity frout.}
		\label{fig:23-frout}
	\end{subfigure}
	\hfill
	\caption{Supply and demand of different commodities for the prediction method poly.}
	\label{fig:23-out}
\end{figure}

\begin{figure}[H]
	\centering
	\begin{subfigure}[t]{0.45\textwidth}
		\centering
		\includegraphics[width=\linewidth]{23-figures/0-poly-lwrstorageout.png} 
		\caption{Commodity lwrstorageout.}
		\label{fig:23-lwrstorageout}
	\end{subfigure}
	\vspace{1cm}
	\begin{subfigure}[t]{0.45\textwidth}
		\centering
		\includegraphics[width=\linewidth]{23-figures/0-poly-frstorageout.png} 
		\caption{Commodity frstorageout.}
		\label{fig:23-frstorageout}
	\end{subfigure}
	\hfill
	\caption{Supply and demand of different commodities for the prediction method poly.}
	\label{fig:23-storageout}
\end{figure}

\begin{figure}[H]
	\centering
	\begin{subfigure}[t]{0.45\textwidth}
		\centering
		\includegraphics[width=\linewidth]{23-figures/0-poly-lwrpu.png} 
		\caption{Commodity lwrpu.}
		\label{fig:23-lwrpu}
	\end{subfigure}
	\vspace{1cm}
	\begin{subfigure}[t]{0.45\textwidth}
		\centering
		\includegraphics[width=\linewidth]{23-figures/0-poly-frpu.png} 
		\caption{Commodity frpu.}
		\label{fig:23-frpu}
	\end{subfigure}
	\hfill
	\caption{Supply and demand of different commodities for the prediction method poly.}
	\label{fig:23-pu}
\end{figure}

\begin{figure}[H]
	\centering
	\begin{subfigure}[t]{0.45\textwidth}
		\centering
		\includegraphics[width=\linewidth]{23-figures/0-poly-lwrreprocessingwaste.png} 
		\caption{Commodity lwrreprocessingwaste.}
		\label{fig:23-lwrreprocessingwaste}
	\end{subfigure}
	\vspace{1cm}
	\begin{subfigure}[t]{0.45\textwidth}
		\centering
		\includegraphics[width=\linewidth]{23-figures/0-poly-frreprocessingwaste.png} 
		\caption{Commodity frreprocessingwaste.}
		\label{fig:23-frreprocessingwaste}
	\end{subfigure}
	\hfill
	\caption{Supply and demand of different commodities for the prediction method poly.}
	\label{fig:23-waste}
\end{figure}

\begin{table}[H]
	\centering
	\caption{Undersupply and oversupply of different commodities using poly to calculate EG01-EG23.}
	\label{tab:23-commod}
	\begin{tabularx}{\textwidth}{lRRR}
		\hline
		Commodity & Undersupplied & Cumulative  & Cumulative \\
		& Timesteps     & Undersupply [$10^3$Kg]  & Oversupply [$10^6$Kg] \\ \hline
		sourceout & 2 & 4713.  &  35648   \\ 
		enrichmentout & 4 & 48.5$x 10^3$  &  42259   \\ 
		lwrout & 1 & 149 & - \\
		frout & 1 & 211 & - \\
		lwrstorageout & 1 & 149 & - \\
		frstorageout & 1 & 211 & - \\
        lwrpu & 1 & 1.7 & - \\
        frpu & 1 & 27.4 & - \\ \hline
	\end{tabularx}
\end{table}

\section{Eg01-Eg24}

Figure \ref{fig:24flow} shows the flow of Eg01-Eg24.

\begin{figure}[H]
	\centering
	\includegraphics[width=\textwidth]{24-figures/24flow.pdf} 
	\hfill
	\caption{EG01-EG24.}
	\label{fig:24flow}
\end{figure}

\subsection{Flat Power Demand}

This section presents plots of power for all the prediction methods. The power demand is 60000 MW throughout the whole simulation. The input files use the installed capacity feature. Buffer is set to zero, back steps is set to two, and steps takes the default value of one.
Figures \ref{fig:24-NO}, \ref{fig:24-DO}, and \ref{fig:24-SO} display the power supply and demand.
Table \ref{tab:24-power} records the number of steps whit under supply, the cumulative under supply, and the cumulative oversupply.

\begin{figure}[H]
	\centering
	\includegraphics[width=\textwidth]{24-figures/24-power-buffer01.png} 
	\hfill
	\caption{NO algorithms.}
	\label{fig:24-NO}
\end{figure}

\begin{figure}[H]
	\centering
	\includegraphics[width=\textwidth]{24-figures/24-power-buffer02.png} 
	\hfill
	\caption{DO algorithms.}
	\label{fig:24-DO}
\end{figure}

\begin{figure}[H]
	\centering
	\includegraphics[width=\textwidth]{24-figures/24-power-buffer03.png} 
	\hfill
	\caption{SO algorithms.}
	\label{fig:24-SO}
\end{figure}

\begin{table}[H]
	\centering
	\caption{Undersupply and oversupply of Power for the different algorithms used to calculate EG01-EG24.}
	\label{tab:24-power}
	\begin{tabularx}{\textwidth}{lRRR}
		\hline
		Algorithm & Undersupplied & Cumulative  & Cumulative \\
		& Timesteps     & Undersupply [GW.mo]  & Oversupply [GW.mo] \\ \hline
		MA        & 26 	& 306.0 &  907.8   \\ 
		ARMA      & 26 	& 306.0 &  907.8   \\ 
		ARCH      & 26 	& 306.0 &  907.8   \\ 
		POLY      &  6 	& 235.0 &  2820.5  \\ 
		EXP\_SMOOTHING 	& 27 & 366.0 & 907.8 \\ 
		HOLT-WINTERS  	& 27 & 366.0 & 907.8 \\ 
		FFT       & 8	& 307.0	& 2820.5 \\ 
		SW\_SEASONAL    & 36 & 308.0 & 398.1	\\ \hline
	\end{tabularx}
\end{table}

\subsection{Linearly increasing Power Demand}

This section presents plots of power for all the prediction methods. The power demand increases linearly with the expression $60000 MW + 250*t MW/year$. The input files use the installed capacity feature. Buffer is set to zero, back steps is set to two, and steps takes the default value of one.
Figures \ref{fig:24-lin-NO}, \ref{fig:24-lin-DO}, and \ref{fig:24-lin-SO} display the power supply and demand.
Table \ref{tab:24-lin-power} records the number of steps whit under supply, the cumulative under supply, and the cumulative oversupply.

\begin{figure}[H]
	\centering
	\includegraphics[width=\textwidth]{24-figures/lin-24-power-buffer01.png} 
	\hfill
	\caption{NO algorithms.}
	\label{fig:24-lin-NO}
\end{figure}

\begin{figure}[H]
	\centering
	\includegraphics[width=\textwidth]{24-figures/lin-24-power-buffer02.png} 
	\hfill
	\caption{DO algorithms.}
	\label{fig:24-lin-DO}
\end{figure}

\begin{figure}[H]
	\centering
	\includegraphics[width=\textwidth]{24-figures/lin-24-power-buffer03.png} 
	\hfill
	\caption{SO algorithms.}
	\label{fig:24-lin-SO}
\end{figure}

\begin{table}[H]
	\centering
	\caption{Undersupply and oversupply of Power for the different algorithms used to calculate EG01-EG24.}
	\label{tab:24-lin-power}
	\begin{tabularx}{\textwidth}{lRRR}
		\hline
		Algorithm & Undersupplied & Cumulative  & Cumulative \\
		& Timesteps     & Undersupply [GW.mo]  & Oversupply [GW.mo] \\ \hline
		MA        & 36 	& 313.7 & 840.9 \\ 
		ARMA      & 36 	& 313.7 & 840.9 \\ 
		ARCH      & 36 	& 316.8 & 859.0 \\ 
		POLY      &  65 & 282.4 & 1974.7 \\ 
		EXP\_SMOOTHING 	& 37 & 373.4 & 828.7 \\ 
		HOLT-WINTERS  	& 37 & 373.4 & 828.7 \\ 
		FFT       & 20	& 315.1	& 2019.1 \\ 
		SW\_SEASONAL    & 107 & 318.8 & 579.09 \\ \hline
	\end{tabularx}
\end{table}

\section{Eg01-Eg29}

Figure \ref{fig:29flow} shows the flow of Eg01-Eg29.

\begin{figure}[H]
	\centering
	\includegraphics[width=\textwidth]{29-figures/29flow.pdf} 
	\hfill
	\caption{EG01-EG29.}
	\label{fig:29flow}
\end{figure}

\subsection{Power}

This section presents plots of power for all the prediction methods. The power demand is 60000 MW throughout the whole simulation. The input files use the installed capacity feature. Buffer is set to zero, back steps is set to two, and steps takes the default value of one.
Figures \ref{fig:29-NO}, \ref{fig:29-DO}, and \ref{fig:29-SO} display the power supply and demand.
Table \ref{tab:29-power} records the number of steps with under supply, the cumulative under supply, and the cumulative oversupply.

\begin{figure}[H]
	\centering
	\includegraphics[width=\textwidth]{29-figures/29-power0-buffer01.png} 
	\hfill
	\caption{NO algorithms.}
	\label{fig:29-NO}
\end{figure}

\begin{figure}[H]
	\centering
	\includegraphics[width=\textwidth]{29-figures/29-power0-buffer02.png} 
	\hfill
	\caption{DO algorithms.}
	\label{fig:29-DO}
\end{figure}

\begin{figure}[H]
	\centering
	\includegraphics[width=\textwidth]{29-figures/29-power0-buffer03.png} 
	\hfill
	\caption{SO algorithms.}
	\label{fig:29-SO}
\end{figure}

\begin{table}[H]
	\centering
	\caption{Undersupply and oversupply of Power for the different algorithms used to calculate EG01-EG29.}
	\label{tab:29-power}
	\begin{tabularx}{\textwidth}{lRRR}
		\hline
		Algorithm & Undersupplied & Cumulative  & Cumulative \\
		& Timesteps     & Undersupply [GW.mo]  & Oversupply [GW.mo] \\ \hline
		MA        & 15 	& 145.0 & 1847.0 \\ 
		ARMA      & 15 	& 145.0 & 1847.0 \\ 
		ARCH      & 15 	& 145.0 & 1846.9 \\ 
		POLY      &  4 	& 90.0 & 4720.3 \\ 
		EXP\_SMOOTHING 	& 16 & 205.0 & 1847.0 \\ 
		HOLT-WINTERS  	& 16 & 205.0 & 1847.0 \\ 
		FFT       &  5	& 150.0	& 4898.0 \\ 
		SW\_SEASONAL    & 14 & 139.0 & 798.9 \\ \hline
	\end{tabularx}
\end{table}

\subsection{Buffer}

This section presents a sensitivity analysis for different values of the buffer. Figure \ref{fig:29-buff} shows a comparison of the cumulative under supply for different buffer sizes.

Figures \ref{fig:29-buf-ma} to \ref{fig:29-buf-fft} display a comparison for some of the methods of the power supply for different buffer sizes.

The input files use the installed capacity feature. Buffer takes the values 0, 2000, 4000, 6000, and 8000. Back steps is set to two, and steps takes the default value of one.

\begin{figure}[H]
	\centering
	\includegraphics[width=\textwidth]{29-figures/29-sens-buffer.png} 
	\hfill
	\caption{Sensitivity analysis for different buffer sizes for some prediction algorithms.}
	\label{fig:29-buff}
\end{figure}

\begin{figure}[H]
	\centering
	\includegraphics[width=\textwidth]{29-figures/29-power-buffer-ma.png} 
	\hfill
	\caption{Power supply for different buffer sizes using ma.}
	\label{fig:29-buf-ma}
\end{figure}

\begin{figure}[H]
	\centering
	\includegraphics[width=\textwidth]{29-figures/29-power-buffer-arma.png} 
	\hfill
	\caption{Power supply for different buffer sizes using arma.}
	\label{fig:29-buf-arma}
\end{figure}

\begin{figure}[H]
	\centering
	\includegraphics[width=\textwidth]{29-figures/29-power-buffer-arch.png} 
	\hfill
	\caption{Power supply for different buffer sizes using arch.}
	\label{fig:29-buf-arch}
\end{figure}

\begin{figure}[H]
	\centering
	\includegraphics[width=\textwidth]{29-figures/29-power-buffer-poly.png} 
	\hfill
	\caption{Power supply for different buffer sizes using poly.}
	\label{fig:29-buf-poly}
\end{figure}

\begin{figure}[H]
	\centering
	\includegraphics[width=\textwidth]{29-figures/29-power-buffer-exp_smoothing.png} 
	\hfill
	\caption{Power supply for different buffer sizes using exp\_smoothing.}
	\label{fig:29-buf-exp_smoothing}
\end{figure}

\begin{figure}[H]
	\centering
	\includegraphics[width=\textwidth]{29-figures/29-power-buffer-holt_winters.png} 
	\hfill
	\caption{Power supply for different buffer sizes using holt\_winters.}
	\label{fig:29-buf-hots_winters}
\end{figure}

\begin{figure}[H]
	\centering
	\includegraphics[width=\textwidth]{29-figures/29-power-buffer-fft.png} 
	\hfill
	\caption{Power supply for different buffer sizes using fft.}
	\label{fig:29-buf-fft}
\end{figure}

\subsection{Steps forward}

This section presents a sensitivity analysis for different values of steps.
Figure \ref{fig:29-steps} shows a comparison of the cumulative under supply for different values of steps forward.

Figures \ref{fig:29-ste-ma} to \ref{fig:29-ste-fft} display a comparison for some of the methods of the power supply for different steps.

The input files use the installed capacity feature. Buffer is set to zero, back steps is set to two, and steps takes the values 1, 2, 3, 4, and 5.

\begin{figure}[H]
	\centering
	\includegraphics[width=\textwidth]{29-figures/29-sens-steps.png} 
	\hfill
	\caption{Sensitivity analysis for different number of steps forward for some prediction algorithms.}
	\label{fig:29-steps}
\end{figure}

\begin{figure}[H]
	\centering
	\includegraphics[width=\textwidth]{29-figures/29-power-buffer0-ma-steps.png} 
	\hfill
	\caption{Power supply for different values of steps forward using ma.}
	\label{fig:29-ste-ma}
\end{figure}

\begin{figure}[H]
	\centering
	\includegraphics[width=\textwidth]{29-figures/29-power-buffer0-arma-steps.png} 
	\hfill
	\caption{Power supply for different values of steps forward using arma.}
	\label{fig:29-ste-arma}
\end{figure}

\begin{figure}[H]
	\centering
	\includegraphics[width=\textwidth]{29-figures/29-power-buffer0-arch-steps.png} 
	\hfill
	\caption{Power supply for different values of steps forward using arch.}
	\label{fig:29-ste-arch}
\end{figure}

\begin{figure}[H]
	\centering
	\includegraphics[width=\textwidth]{29-figures/29-power-buffer0-poly-steps.png} 
	\hfill
	\caption{Power supply for different values of steps forward using poly.}
	\label{fig:29-ste-poly}
\end{figure}

\begin{figure}[H]
	\centering
	\includegraphics[width=\textwidth]{29-figures/29-power-buffer0-exp_smoothing-steps.png} 
	\hfill
	\caption{Power supply for different values of steps forward using exp\_smoothing.}
	\label{fig:29-ste-exp_smoothing}
\end{figure}

\begin{figure}[H]
	\centering
	\includegraphics[width=\textwidth]{29-figures/29-power-buffer0-holt_winters-steps.png} 
	\hfill
	\caption{Power supply for different values of steps forward using holt\_winters.}
	\label{fig:29-ste-hots_winters}
\end{figure}

\begin{figure}[H]
	\centering
	\includegraphics[width=\textwidth]{29-figures/29-power-buffer0-fft-steps.png} 
	\hfill
	\caption{Power supply for different values of steps forward using fft.}
	\label{fig:29-ste-fft}
\end{figure}

\subsection{Different commodities}

Table \ref{tab:29-commodities} shows the name of the variables used in the simulations and what they represent in the cycle. Table \ref{tab:29-commod} presents the number of steps of under supply, cumulative under supply, and cumulative oversupply for some of the commodities.

In this simulation, front-end commodities are power, sourceout, enrichmentout, frmixerout, and moxmixerout. All the rest of the commodities are back-end.

\begin{table}[H]
	\centering
	\caption{Commodity names used in the simulation of EG01-EG29.}
	\label{tab:29-commodities}
	\begin{tabularx}{\textwidth}{lLL}
		\hline
		Commodity name  & Figures & Represents \\ \hline
		power           & \ref{fig:29-power} & Power \\
		sourceout       & \ref{fig:29-sourceout} & Natural-U \\
		enrichmentout   & \ref{fig:29-enrichmentout} & Enriched-U \\
		frmixerout      &  \ref{fig:29-frmixerout} & FR fuel \\
    	moxmixerout     &  \ref{fig:29-moxmixerout} & MOX fuel \\
		lwrout          & \ref{fig:29-out1} & Spent fuel of LWRs \\
		frout           & \ref{fig:29-frout} & Spent fuel of FRs \\
		frout           & \ref{fig:29-moxout} & Spent fuel of MOX LWRs \\
		lwrstorageout   & \ref{fig:29-storageout1} & Cooled down spent fuel of LWRs \\
		frstorageout    & \ref{fig:29-frstorageout} & Cooled down spent fuel of FRs \\	
		moxstorageout    & \ref{fig:29-moxstorageout} & Cooled down spent fuel of MOX LWRs \\	
		lwrpu   & \ref{fig:29-pu1} & Pu from spent fuel of LWRs \\
		frpu    & \ref{fig:29-frpu} & Pu from spent fuel of FRs \\
		moxpu    & \ref{fig:29-moxpu} & Pu from spent fuel of MOX LWRs \\
		lwrreprocessingwaste & \ref{fig:29-lwrreprocessingwaste} & Waste from the reprocessing of spent fuel of LWRs \\
		frreprocessingwaste & \ref{fig:29-frreprocessingwaste} & Waste from the reprocessing of spent fuel of FRs \\
		moxreprocessingwaste & \ref{fig:29-moxreprocessingwaste} & Waste from the reprocessing of spent fuel of MOX LWRs \\ \hline
			
	\end{tabularx}
\end{table}

\begin{figure}[H]
	\centering
	\includegraphics[width=\textwidth]{29-figures/0-poly-power.png} 
	\hfill
	\caption{Supply and demand of the commodity power.}
	\label{fig:29-power}
\end{figure}

\begin{figure}[H]
	\centering
	\begin{subfigure}[]{0.45\textwidth}
		\centering
		\includegraphics[width=\linewidth]{29-figures/0-poly-sourceout.png} 
		\caption{Commodity sourceout.}
		\label{fig:29-sourceout}
	\end{subfigure}
	\vspace{1cm}
	\begin{subfigure}[]{0.45\textwidth}
		\centering
		\includegraphics[width=\linewidth]{29-figures/0-poly-enrichmentout.png} 
		\caption{Commodity enrichmentout.}
		\label{fig:29-enrichmentout}
	\end{subfigure}
	\hfill
	\caption{Supply and demand of different commodities for the prediction method poly.}
	\label{fig:29-front}
\end{figure}

\begin{figure}[H]
	\centering
	\begin{subfigure}[]{0.45\textwidth}
		\centering
		\includegraphics[width=\linewidth]{29-figures/0-poly-frmixerout.png} 
		\caption{Commodity frmixerout.}
		\label{fig:29-frmixerout}
	\end{subfigure}
	\vspace{1cm}
	\begin{subfigure}[]{0.45\textwidth}
		\centering
		\includegraphics[width=\linewidth]{29-figures/0-poly-moxmixerout.png} 
		\caption{Commodity moxmixerout.}
		\label{fig:29-moxmixerout}
	\end{subfigure}
	\hfill
	\caption{Supply and demand of different commodities for the prediction method poly.}
	\label{fig:29-mix}
\end{figure}

\begin{figure}[H]
	\centering
	\includegraphics[width=\textwidth]{29-figures/0-poly-lwrout.png} 
	\hfill
	\caption{Supply and demand of the commodity lwrout.}
	\label{fig:29-out1}
\end{figure}

\begin{figure}[H]
	\centering
	\begin{subfigure}[]{0.45\textwidth}
		\centering
		\includegraphics[width=\linewidth]{29-figures/0-poly-frout.png} 
		\caption{Commodity frout.}
		\label{fig:29-frout}
	\end{subfigure}
	\vspace{1cm}
	\begin{subfigure}[]{0.45\textwidth}
		\centering
		\includegraphics[width=\linewidth]{29-figures/0-poly-moxout.png} 
		\caption{Commodity moxout.}
		\label{fig:29-moxout}
	\end{subfigure}
	\hfill
	\caption{Supply and demand of different commodities for the prediction method poly.}
	\label{fig:29-out2}
\end{figure}

\begin{figure}[H]
	\centering
	\includegraphics[width=\textwidth]{29-figures/0-poly-lwrstorageout.png} 
	\hfill
	\caption{Supply and demand of the commodity lwrstorageout.}
	\label{fig:29-storageout1}
\end{figure}

\begin{figure}[H]
	\centering
	\begin{subfigure}[t]{0.45\textwidth}
		\centering
		\includegraphics[width=\linewidth]{29-figures/0-poly-frstorageout.png} 
		\caption{Commodity frstorageout.}
		\label{fig:29-frstorageout}
	\end{subfigure}
	\vspace{1cm}
	\begin{subfigure}[t]{0.45\textwidth}
		\centering
		\includegraphics[width=\linewidth]{29-figures/0-poly-moxstorageout.png} 
		\caption{Commodity moxstorageout.}
		\label{fig:29-moxstorageout}
	\end{subfigure}
	\hfill
	\caption{Supply and demand of different commodities for the prediction method poly.}
	\label{fig:29-storageout2}
\end{figure}

\begin{figure}[H]
	\centering
	\includegraphics[width=\textwidth]{29-figures/0-poly-lwrpu.png} 
	\hfill
	\caption{Supply and demand of the commodity lwrpu.}
	\label{fig:29-pu1}
\end{figure}

\begin{figure}[H]
	\centering
	\begin{subfigure}[t]{0.45\textwidth}
		\centering
		\includegraphics[width=\linewidth]{23-figures/0-poly-frpu.png} 
		\caption{Commodity frpu.}
		\label{fig:29-frpu}
	\end{subfigure}
	\vspace{1cm}
	\begin{subfigure}[t]{0.45\textwidth}
		\centering
		\includegraphics[width=\linewidth]{29-figures/0-poly-moxpu.png} 
		\caption{Commodity moxpu.}
		\label{fig:29-moxpu}
	\end{subfigure}
	\hfill
	\caption{Supply and demand of different commodities for the prediction method poly.}
	\label{fig:29-pu2}
\end{figure}

\begin{figure}[H]
	\centering
	\includegraphics[width=\textwidth]{29-figures/0-poly-lwrreprocessingwaste.png} 
	\hfill
	\caption{Supply and demand of the commodity lwrreprocessingwaste.}
	\label{fig:29-lwrreprocessingwaste}
\end{figure}

\begin{figure}[H]
	\centering
	\begin{subfigure}[t]{0.45\textwidth}
		\centering
		\includegraphics[width=\linewidth]{23-figures/0-poly-frreprocessingwaste.png} 
		\caption{Commodity frreprocessingwaste.}
		\label{fig:29-frreprocessingwaste}
	\end{subfigure}
	\vspace{1cm}
	\begin{subfigure}[t]{0.45\textwidth}
		\centering
		\includegraphics[width=\linewidth]{29-figures/0-poly-moxreprocessingwaste.png} 
		\caption{Commodity moxreprocessingwaste.}
		\label{fig:29-moxreprocessingwaste}
	\end{subfigure}
	\hfill
	\caption{Supply and demand of different commodities for the prediction method poly.}
	\label{fig:29-waste}
\end{figure}

\begin{table}[H]
	\centering
	\caption{Undersupply and oversupply of different commodities using poly to calculate EG01-EG29.}
	\label{tab:29-commod}
	\begin{tabularx}{\textwidth}{lRRR}
		\hline
		Commodity & Undersupplied & Cumulative  & Cumulative \\
		& Timesteps & Undersupply [$10^3$Kg]  & Oversupply [$10^6$Kg] \\ \hline
		sourceout & 1 & 34394.0  & 132319869.5 \\ 
		enrichmentout & 1 & 16126.1 & 102751545581.6 \\

		lwrout & 1 & 1791.8 & - \\
		frout & 1 & 142.2 & - \\
		moxout & 1 & 265.0 & - \\

		frmixerout & 2 & 284.4 & 124827.3 \\
        moxmixerout & 2 & 530.1 & 354541.5 \\

		lwrstorageout & 1 & 1791.8 & - \\
		frstorageout & 1 & 142.2 & - \\
		moxstorageout & 1 & 265.0 & - \\ \hline
	\end{tabularx}
\end{table}

\section{Eg01-Eg30}

Figure \ref{fig:30flow} shows the flow of Eg01-Eg30.

\begin{figure}[H]
	\centering
	\includegraphics[width=\textwidth]{30-figures/30flow.pdf} 
	\hfill
	\caption{EG01-EG30.}
	\label{fig:30flow}
\end{figure}

\subsection{Flat Power Demand}

This section presents plots of power for all the prediction methods. The power demand is 60000 MW throughout the whole simulation. The input files use the installed capacity feature. Buffer is set to zero, back steps is set to two, and steps takes the default value of one.
Figures \ref{fig:30-NO}, \ref{fig:30-DO}, and \ref{fig:30-SO} display the power supply and demand.
Table \ref{tab:30-power} records the number of steps whit under supply, the cumulative under supply, and the cumulative oversupply.

\begin{figure}[H]
	\centering
	\includegraphics[width=\textwidth]{30-figures/30-power-buffer01.png} 
	\hfill
	\caption{NO algorithms.}
	\label{fig:30-NO}
\end{figure}

\begin{figure}[H]
	\centering
	\includegraphics[width=\textwidth]{30-figures/30-power-buffer02.png} 
	\hfill
	\caption{DO algorithms.}
	\label{fig:30-DO}
\end{figure}

\begin{figure}[H]
	\centering
	\includegraphics[width=\textwidth]{30-figures/30-power-buffer03.png} 
	\hfill
	\caption{SO algorithms.}
	\label{fig:30-SO}
\end{figure}

\begin{table}[H]
	\centering
	\caption{Undersupply and oversupply of Power for the different algorithms used to calculate EG01-EG24.}
	\label{tab:30-power}
	\begin{tabularx}{\textwidth}{lRRR}
		\hline
		Algorithm & Undersupplied & Cumulative  & Cumulative \\
		& Timesteps     & Undersupply [GW.mo]  & Oversupply [GW.mo] \\ \hline
		MA        & 15 & 144.0 & 1718.7 \\ 
		ARMA      & 15 & 144.0 & 1718.7 \\ 
		ARCH      & 15 & 144.0 & 1718.7 \\ 
		POLY      & 4  & 90.0 & 5026.5 \\ 
		EXP\_SMOOTHING 	& 16 & 204.0 & 1718.7 \\ 
		HOLT-WINTERS  	& 16 & 204.0 & 1718.7 \\ 
		FFT       & 5 & 150.0 & 5044.5 \\ 
		SW\_SEASONAL    & 14 & 141.0 & 784.0 \\ \hline
	\end{tabularx}
\end{table}

\subsection{Linearly increasing Power Demand}

This section presents plots of power for all the prediction methods. The power demand increases linearly with the expression $60000 MW + 250*t MW/year$. The input files use the installed capacity feature. Buffer is set to zero, back steps is set to two, and steps takes the default value of one.
Figures \ref{fig:30-lin-NO}, \ref{fig:30-lin-DO}, and \ref{fig:30-lin-SO} display the power supply and demand.
Table \ref{tab:30-lin-power} records the number of steps whit under supply, the cumulative under supply, and the cumulative oversupply.

\begin{figure}[H]
	\centering
	\includegraphics[width=\textwidth]{30-figures/lin-30-power-buffer01.png} 
	\hfill
	\caption{NO algorithms.}
	\label{fig:30-lin-NO}
\end{figure}

\begin{figure}[H]
	\centering
	\includegraphics[width=\textwidth]{30-figures/lin-30-power-buffer02.png} 
	\hfill
	\caption{DO algorithms.}
	\label{fig:30-lin-DO}
\end{figure}

\begin{figure}[H]
	\centering
	\includegraphics[width=\textwidth]{30-figures/lin-30-power-buffer03.png} 
	\hfill
	\caption{SO algorithms.}
	\label{fig:30-lin-SO}
\end{figure}

\begin{table}[H]
	\centering
	\caption{Undersupply and oversupply of Power for the different algorithms used to calculate EG01-EG24.}
	\label{tab:30-lin-power}
	\begin{tabularx}{\textwidth}{lRRR}
		\hline
		Algorithm & Undersupplied & Cumulative  & Cumulative \\
		& Timesteps     & Undersupply [GW.mo]  & Oversupply [GW.mo] \\ \hline
		MA        & 24 & 152.3 & 1334.1 \\ 
		ARMA      & 24 & 152.3 & 1334.1 \\ 
		ARCH      & 21 & 152.1 & 1355.9 \\ 
		POLY      &  9 & 92.5 & 3073.1 \\ 
		EXP\_SMOOTHING 	& 25 & 211.6 & 1317.8 \\ 
		HOLT-WINTERS  	& 25 & 211.6 & 1317.8 \\ 
		FFT       & 9 & 152.5 & 3079.4 \\ 
		SW\_SEASONAL  & 51 & 147.3 & 873.4 \\ \hline
	\end{tabularx}
\end{table}

\end{document}